
%% bare_conf.tex
%% V1.3
%% 2007/01/11
%% by Michael Shell
%% See:
%% http://www.michaelshell.org/
%% for current contact information.
%%
%% This is a skeleton file demonstrating the use of IEEEtran.cls
%% (requires IEEEtran.cls version 1.7 or later) with an IEEE conference paper.
%%
%% Support sites:
%% http://www.michaelshell.org/tex/ieeetran/
%% http://www.ctan.org/tex-archive/macros/latex/contrib/IEEEtran/
%% and
%% http://www.ieee.org/

%%*************************************************************************
%% Legal Notice:
%% This code is offered as-is without any warranty either expressed or
%% implied; without even the implied warranty of MERCHANTABILITY or
%% FITNESS FOR A PARTICULAR PURPOSE! 
%% User assumes all risk.
%% In no event shall IEEE or any contributor to this code be liable for
%% any damages or losses, including, but not limited to, incidental,
%% consequential, or any other damages, resulting from the use or misuse
%% of any information contained here.
%%
%% All comments are the opinions of their respective authors and are not
%% necessarily endorsed by the IEEE.
%%
%% This work is distributed under the LaTeX Project Public License (LPPL)
%% ( http://www.latex-project.org/ ) version 1.3, and may be freely used,
%% distributed and modified. A copy of the LPPL, version 1.3, is included
%% in the base LaTeX documentation of all distributions of LaTeX released
%% 2003/12/01 or later.
%% Retain all contribution notices and credits.
%% ** Modified files should be clearly indicated as such, including  **
%% ** renaming them and changing author support contact information. **
%%
%% File list of work: IEEEtran.cls, IEEEtran_HOWTO.pdf, bare_adv.tex,
%%                    bare_conf.tex, bare_jrnl.tex, bare_jrnl_compsoc.tex
%%*************************************************************************

% *** Authors should verify (and, if needed, correct) their LaTeX system  ***
% *** with the testflow diagnostic prior to trusting their LaTeX platform ***
% *** with production work. IEEE's font choices can trigger bugs that do  ***
% *** not appear when using other class files.                            ***
% The testflow support page is at:
% http://www.michaelshell.org/tex/testflow/



% Note that the a4paper option is mainly intended so that authors in
% countries using A4 can easily print to A4 and see how their papers will
% look in print - the typesetting of the document will not typically be
% affected with changes in paper size (but the bottom and side margins will).
% Use the testflow package mentioned above to verify correct handling of
% both paper sizes by the user's LaTeX system.
%
% Also note that the "draftcls" or "draftclsnofoot", not "draft", option
% should be used if it is desired that the figures are to be displayed in
% draft mode.
%
\documentclass[10pt, conference, compsocconf]{IEEEtran}
% Add the compsocconf option for Computer Society conferences.
%
% If IEEEtran.cls has not been installed into the LaTeX system files,
% manually specify the path to it like:
% \documentclass[conference]{../sty/IEEEtran}





% Some very useful LaTeX packages include:
% (uncomment the ones you want to load)


% *** MISC UTILITY PACKAGES ***
%
%\usepackage{ifpdf}
% Heiko Oberdiek's ifpdf.sty is very useful if you need conditional
% compilation based on whether the output is pdf or dvi.
% usage:
% \ifpdf
%   % pdf code
% \else
%   % dvi code
% \fi
% The latest version of ifpdf.sty can be obtained from:
% http://www.ctan.org/tex-archive/macros/latex/contrib/oberdiek/
% Also, note that IEEEtran.cls V1.7 and later provides a builtin
% \ifCLASSINFOpdf conditional that works the same way.
% When switching from latex to pdflatex and vice-versa, the compiler may
% have to be run twice to clear warning/error messages.






% *** CITATION PACKAGES ***
%
%\usepackage{cite}
% cite.sty was written by Donald Arseneau
% V1.6 and later of IEEEtran pre-defines the format of the cite.sty package
% \cite{} output to follow that of IEEE. Loading the cite package will
% result in citation numbers being automatically sorted and properly
% "compressed/ranged". e.g., [1], [9], [2], [7], [5], [6] without using
% cite.sty will become [1], [2], [5]--[7], [9] using cite.sty. cite.sty's
% \cite will automatically add leading space, if needed. Use cite.sty's
% noadjust option (cite.sty V3.8 and later) if you want to turn this off.
% cite.sty is already installed on most LaTeX systems. Be sure and use
% version 4.0 (2003-05-27) and later if using hyperref.sty. cite.sty does
% not currently provide for hyperlinked citations.
% The latest version can be obtained at:
% http://www.ctan.org/tex-archive/macros/latex/contrib/cite/
% The documentation is contained in the cite.sty file itself.






% *** GRAPHICS RELATED PACKAGES ***
%
\ifCLASSINFOpdf
  % \usepackage[pdftex]{graphicx}
  % declare the path(s) where your graphic files are
  % \graphicspath{{../pdf/}{../jpeg/}}
  % and their extensions so you won't have to specify these with
  % every instance of \includegraphics
  % \DeclareGraphicsExtensions{.pdf,.jpeg,.png}
\else
  % or other class option (dvipsone, dvipdf, if not using dvips). graphicx
  % will default to the driver specified in the system graphics.cfg if no
  % driver is specified.
  % \usepackage[dvips]{graphicx}
  % declare the path(s) where your graphic files are
  % \graphicspath{{../eps/}}
  % and their extensions so you won't have to specify these with
  % every instance of \includegraphics
  % \DeclareGraphicsExtensions{.eps}
\fi
% graphicx was written by David Carlisle and Sebastian Rahtz. It is
% required if you want graphics, photos, etc. graphicx.sty is already
% installed on most LaTeX systems. The latest version and documentation can
% be obtained at: 
% http://www.ctan.org/tex-archive/macros/latex/required/graphics/
% Another good source of documentation is "Using Imported Graphics in
% LaTeX2e" by Keith Reckdahl which can be found as epslatex.ps or
% epslatex.pdf at: http://www.ctan.org/tex-archive/info/
%
% latex, and pdflatex in dvi mode, support graphics in encapsulated
% postscript (.eps) format. pdflatex in pdf mode supports graphics
% in .pdf, .jpeg, .png and .mps (metapost) formats. Users should ensure
% that all non-photo figures use a vector format (.eps, .pdf, .mps) and
% not a bitmapped formats (.jpeg, .png). IEEE frowns on bitmapped formats
% which can result in "jaggedy"/blurry rendering of lines and letters as
% well as large increases in file sizes.
%
% You can find documentation about the pdfTeX application at:
% http://www.tug.org/applications/pdftex





% *** MATH PACKAGES ***
%
%\usepackage[cmex10]{amsmath}
% A popular package from the American Mathematical Society that provides
% many useful and powerful commands for dealing with mathematics. If using
% it, be sure to load this package with the cmex10 option to ensure that
% only type 1 fonts will utilized at all point sizes. Without this option,
% it is possible that some math symbols, particularly those within
% footnotes, will be rendered in bitmap form which will result in a
% document that can not be IEEE Xplore compliant!
%
% Also, note that the amsmath package sets \interdisplaylinepenalty to 10000
% thus preventing page breaks from occurring within multiline equations. Use:
%\interdisplaylinepenalty=2500
% after loading amsmath to restore such page breaks as IEEEtran.cls normally
% does. amsmath.sty is already installed on most LaTeX systems. The latest
% version and documentation can be obtained at:
% http://www.ctan.org/tex-archive/macros/latex/required/amslatex/math/





% *** SPECIALIZED LIST PACKAGES ***
%
%\usepackage{algorithmic}
% algorithmic.sty was written by Peter Williams and Rogerio Brito.
% This package provides an algorithmic environment fo describing algorithms.
% You can use the algorithmic environment in-text or within a figure
% environment to provide for a floating algorithm. Do NOT use the algorithm
% floating environment provided by algorithm.sty (by the same authors) or
% algorithm2e.sty (by Christophe Fiorio) as IEEE does not use dedicated
% algorithm float types and packages that provide these will not provide
% correct IEEE style captions. The latest version and documentation of
% algorithmic.sty can be obtained at:
% http://www.ctan.org/tex-archive/macros/latex/contrib/algorithms/
% There is also a support site at:
% http://algorithms.berlios.de/index.html
% Also of interest may be the (relatively newer and more customizable)
% algorithmicx.sty package by Szasz Janos:
% http://www.ctan.org/tex-archive/macros/latex/contrib/algorithmicx/




% *** ALIGNMENT PACKAGES ***
%
%\usepackage{array}
% Frank Mittelbach's and David Carlisle's array.sty patches and improves
% the standard LaTeX2e array and tabular environments to provide better
% appearance and additional user controls. As the default LaTeX2e table
% generation code is lacking to the point of almost being broken with
% respect to the quality of the end results, all users are strongly
% advised to use an enhanced (at the very least that provided by array.sty)
% set of table tools. array.sty is already installed on most systems. The
% latest version and documentation can be obtained at:
% http://www.ctan.org/tex-archive/macros/latex/required/tools/


%\usepackage{mdwmath}
%\usepackage{mdwtab}
% Also highly recommended is Mark Wooding's extremely powerful MDW tools,
% especially mdwmath.sty and mdwtab.sty which are used to format equations
% and tables, respectively. The MDWtools set is already installed on most
% LaTeX systems. The lastest version and documentation is available at:
% http://www.ctan.org/tex-archive/macros/latex/contrib/mdwtools/


% IEEEtran contains the IEEEeqnarray family of commands that can be used to
% generate multiline equations as well as matrices, tables, etc., of high
% quality.


%\usepackage{eqparbox}
% Also of notable interest is Scott Pakin's eqparbox package for creating
% (automatically sized) equal width boxes - aka "natural width parboxes".
% Available at:
% http://www.ctan.org/tex-archive/macros/latex/contrib/eqparbox/





% *** SUBFIGURE PACKAGES ***
%\usepackage[tight,footnotesize]{subfigure}
% subfigure.sty was written by Steven Douglas Cochran. This package makes it
% easy to put subfigures in your figures. e.g., "Figure 1a and 1b". For IEEE
% work, it is a good idea to load it with the tight package option to reduce
% the amount of white space around the subfigures. subfigure.sty is already
% installed on most LaTeX systems. The latest version and documentation can
% be obtained at:
% http://www.ctan.org/tex-archive/obsolete/macros/latex/contrib/subfigure/
% subfigure.sty has been superceeded by subfig.sty.



%\usepackage[caption=false]{caption}
%\usepackage[font=footnotesize]{subfig}
% subfig.sty, also written by Steven Douglas Cochran, is the modern
% replacement for subfigure.sty. However, subfig.sty requires and
% automatically loads Axel Sommerfeldt's caption.sty which will override
% IEEEtran.cls handling of captions and this will result in nonIEEE style
% figure/table captions. To prevent this problem, be sure and preload
% caption.sty with its "caption=false" package option. This is will preserve
% IEEEtran.cls handing of captions. Version 1.3 (2005/06/28) and later 
% (recommended due to many improvements over 1.2) of subfig.sty supports
% the caption=false option directly:
%\usepackage[caption=false,font=footnotesize]{subfig}
%
% The latest version and documentation can be obtained at:
% http://www.ctan.org/tex-archive/macros/latex/contrib/subfig/
% The latest version and documentation of caption.sty can be obtained at:
% http://www.ctan.org/tex-archive/macros/latex/contrib/caption/




% *** FLOAT PACKAGES ***
%
%\usepackage{fixltx2e}
% fixltx2e, the successor to the earlier fix2col.sty, was written by
% Frank Mittelbach and David Carlisle. This package corrects a few problems
% in the LaTeX2e kernel, the most notable of which is that in current
% LaTeX2e releases, the ordering of single and double column floats is not
% guaranteed to be preserved. Thus, an unpatched LaTeX2e can allow a
% single column figure to be placed prior to an earlier double column
% figure. The latest version and documentation can be found at:
% http://www.ctan.org/tex-archive/macros/latex/base/



%\usepackage{stfloats}
% stfloats.sty was written by Sigitas Tolusis. This package gives LaTeX2e
% the ability to do double column floats at the bottom of the page as well
% as the top. (e.g., "\begin{figure*}[!b]" is not normally possible in
% LaTeX2e). It also provides a command:
%\fnbelowfloat
% to enable the placement of footnotes below bottom floats (the standard
% LaTeX2e kernel puts them above bottom floats). This is an invasive package
% which rewrites many portions of the LaTeX2e float routines. It may not work
% with other packages that modify the LaTeX2e float routines. The latest
% version and documentation can be obtained at:
% http://www.ctan.org/tex-archive/macros/latex/contrib/sttools/
% Documentation is contained in the stfloats.sty comments as well as in the
% presfull.pdf file. Do not use the stfloats baselinefloat ability as IEEE
% does not allow \baselineskip to stretch. Authors submitting work to the
% IEEE should note that IEEE rarely uses double column equations and
% that authors should try to avoid such use. Do not be tempted to use the
% cuted.sty or midfloat.sty packages (also by Sigitas Tolusis) as IEEE does
% not format its papers in such ways.





% *** PDF, URL AND HYPERLINK PACKAGES ***
%
%\usepackage{url}
% url.sty was written by Donald Arseneau. It provides better support for
% handling and breaking URLs. url.sty is already installed on most LaTeX
% systems. The latest version can be obtained at:
% http://www.ctan.org/tex-archive/macros/latex/contrib/misc/
% Read the url.sty source comments for usage information. Basically,
% \url{my_url_here}.





% *** Do not adjust lengths that control margins, column widths, etc. ***
% *** Do not use packages that alter fonts (such as pslatex).         ***
% There should be no need to do such things with IEEEtran.cls V1.6 and later.
% (Unless specifically asked to do so by the journal or conference you plan
% to submit to, of course. )


% correct bad hyphenation here
\hyphenation{op-tical net-works semi-conduc-tor}

% provided by users
\usepackage{psfrag}
\usepackage{amssymb}
%\usepackage{theorem}
\def\comment#1{}
\usepackage{fancyvrb}
\usepackage{alltt}
\usepackage{listings}
\usepackage{url}
\usepackage[linesnumbered]{algorithm2e}
%\usepackage{program}
%\usepackage{array}
\usepackage{hyperref}
%\usepackage{microtype}
%\usepackage{algorithm}
%\usepackage{algorithmic}

%\def\IEEEbibitemsep{2pt plus .5pt}
\IEEEoverridecommandlockouts


\begin{document}
%
% paper title
% can use linebreaks \\ within to get better formatting as desired
\title{Formal Specification and Model Checking of \\an Autonomous Vehicle
Merging Protocol
\thanks{This work was partially supported by JSPS
    KAKENHI Grant Number JP19H04082.}
%\thanks{DOI reference number: 10.18293/DMSVIVA2021-004 }
}

% author names and affiliations
% use a multiple column layout for up to two different
% affiliations

\author{\IEEEauthorblockN{Minxuan Liu, Dang Duy Bui, Duong Dinh Tran, Kazuhiro Ogata}
\IEEEauthorblockA{School of Information Science\\
Japan Advanced Institute of Science and Technology (JAIST)\\
1-1 Asahidai, Nomi, Ishikawa 923-1292, Japan\\
Email: \{liuminxuan,bddang,duongtd,ogata\}@jaist.ac.jp}
}

%\author{\IEEEauthorblockN{Authors Name/s per 1st Affiliation (Author)}
%\IEEEauthorblockA{line 1 (of Affiliation): dept. name of organization\\
%line 2: name of organization, acronyms acceptable\\
%line 3: City, Country\\
%line 4: Email: name@xyz.com}
%\and
%\IEEEauthorblockN{Authors Name/s per 2nd Affiliation (Author)}
%\IEEEauthorblockA{line 1 (of Affiliation): dept. name of organization\\
%line 2: name of organization, acronyms acceptable\\
%line 3: City, Country\\
%line 4: Email: name@xyz.com}
%}

% conference papers do not typically use \thanks and this command
% is locked out in conference mode. If really needed, such as for
% the acknowledgment of grants, issue a \IEEEoverridecommandlockouts
% after \documentclass

% for over three affiliations, or if they all won't fit within the width
% of the page, use this alternative format:
% 
%\author{\IEEEauthorblockN{Michael Shell\IEEEauthorrefmark{1},
%Homer Simpson\IEEEauthorrefmark{2},
%James Kirk\IEEEauthorrefmark{3}, 
%Montgomery Scott\IEEEauthorrefmark{3} and
%Eldon Tyrell\IEEEauthorrefmark{4}}
%\IEEEauthorblockA{\IEEEauthorrefmark{1}School of Electrical and Computer Engineering\\
%Georgia Institute of Technology,
%Atlanta, Georgia 30332--0250\\ Email: see http://www.michaelshell.org/contact.html}
%\IEEEauthorblockA{\IEEEauthorrefmark{2}Twentieth Century Fox, Springfield, USA\\
%Email: homer@thesimpsons.com}
%\IEEEauthorblockA{\IEEEauthorrefmark{3}Starfleet Academy, San Francisco, California 96678-2391\\
%Telephone: (800) 555--1212, Fax: (888) 555--1212}
%\IEEEauthorblockA{\IEEEauthorrefmark{4}Tyrell Inc., 123 Replicant Street, Los Angeles, California 90210--4321}}




% use for special paper notices
%\IEEEspecialpapernotice{(Invited Paper)}




% make the title area
\maketitle

\begin{abstract}
Self-driving/autonomous vehicles are promising transportation of the
future providing many conveniences for people.  Because safety is a
point of paramount significance for any autonomous vehicle system, a
large number of researches have proposed mechanisms/protocols to
automatically control vehicles running on public roads without any
collision. S. Aoki and R. Rajkumar proposed a merging protocol for
autonomous vehicles. The protocol controls autonomous vehicles running
on two one-way lanes (one through lane and one non-through lane) such
that such vehicles pass through the merge point without any
collision. The protocol uses real-time information, such as the
average speed of multiple vehicles running on a lane. We have revised
the protocol such that it never uses such information because the
seepf of a vehicle may drastically change in a short moment. We have
then formally specified the revised version of the protocol in Maude
and conducted some model checking experiments with Maude that the
revised version enjoys some desired properties.
\end{abstract}

\begin{IEEEkeywords}
autonomous vehicles; Maude; merging protocols; model checking; self-driving vehicles

%merging protocol; a-merging protocol; Maude; model checking
% anti-fairness; fairness; linear temporal logic; liveness property;
% model checking;
\end{IEEEkeywords}

% For peer review papers, you can put extra information on the cover
% page as needed:
% \ifCLASSOPTIONpeerreview
% \begin{center} \bfseries EDICS Category: 3-BBND \end{center}
% \fi
%
% For peerreview papers, this IEEEtran command inserts a page break and
% creates the second title. It will be ignored for other modes.
\IEEEpeerreviewmaketitle


\setlength{\parindent}{1em}
\section{Introduction}
 \label{sect_intro}

In recent years, there is a large amount of attention on building
self-driving/autonomous vehicle systems all around the world.  In
addition to hardware technologies, software technologies behind are
considered as the key point of any autonomous vehicle system.  In
addition to allowing vehicles to autonomously and safely run on a
straight road, it is necessary to allow vehicles to autonomously and
safely change to another lane from the current lane, pass through a
four-way intersection, pass through a merge point of two one-way
lanes, etc. Many researchers have proposed mechanisms/protocols in
order to address these challenges.

S. Aoki and R. Rajkumar~\cite{10.1145/3055004.3055028} proposed a
merging protocol for autonomous vehicles. The protocol controls
autonomous vehicles running on two one-way lanes (one through lane and
one non-through lane) such that such vehicles pass the merge point
without any collision. In the protocol, autonomous vehicles are
assumed to use both vehicle-to-vehicle (V2V) communications and
sensor-based perception systems.  According to the paper, the authors
claim that the protocol is safe and efficient. The protocol is called
the Aoki-Rajkumar merging ptotocol or simply the AR merging protocol.

The AR merging protocol relies on real-time information, such as the
average speed of multiple vehicles running on a lane. We think that
such information is fragile because the speeds of vehicles may
drastically change in a short moment, say by sudden
breaking. Therefore, we would like not to rely on such information. We
have then revised the AR merging protocol. The revised version of the
AR merging protocol is called our merging protocol in the present
paper. We have also formally specified our merging protocol in
Maude~\cite{Clavel2007LNCS}, a rewriting logic-based
specification/programming language. The processor/system of Maude is
also called Maude. Maude is equipped with model checking
facilities. We have conducted model checking experiments that our
merging protocol enjoys some desired properties with Maude. One
desired property is the mutual exclusion property that means that
there is always at most one vehicle in the merge point.  In the
present paper, we describe our merging protocol, how to formally
specify the protocol in Maude and how to model check that the protocol
enjoys some desired properties with Maude.

The rest of this paper is organized as follows. Sect.\,\ref{sect_Prel}
mentions some preliminaries, such as Kripke structure and Maude.
Sect.\,\ref{sect_oriproto} introduces the AR merging protocol.
Sect.\,\ref{sect_reviproto} describes our merging protocol.  We
describe how to formally specify our merging protocol in Maude and how
to model check that the protocol enjoys some desired properties with
Maude in Sect.\,\ref{sect_formal} and Sect.\,\ref{sect_model},
respectively.  Sect.\,\ref{sect_Relate} mentions some related work.
Finally, we conclude the present paper in Sect.\,\ref{concl_sect}.

The specification of the protocol in Maude presented in this paper is
available at
\url{https://github.com/liuminxuan/Specification-of-an-abstract-merging-protocol/}.


\section{Preliminaries}
 \label{sect_Prel}
 
A Kripke structure $K$ is $\langle S,I,T,P,L \rangle$, where $S$ is a set
of states, $I \subseteq S$ is the set of initial states, $T \subseteq S \times S$
is a total binary relation over $S$, $P$ is a set of atomic
propositions and $L$ is a labeling function whose type is
$S \rightarrow 2^P$ . Each element $(s, s') \in T$ is called a state transition
from $s$ to $s'$ and $T$ may be called the state transitions
(with respect to $K$). For a state $s \in S$, $L(s)$ is the set
of atomic propositions that hold in $s$. A path $\pi$ is an infinite
sequence $s_0, \ldots , s_i, s_{i+1}, \ldots$ of states such that $s_i \in S$ and
$(s_i, s_{i+1}) \in T$ for each $i$. Let $\pi^i$ be $s_i, s_{i+1}, \ldots$ and $\pi(i)$ be
$s_i$. Let $P$ be the set of all paths. $\pi$ is called a computation
if $\pi(0) \in I$. Let $C$ be the set of all computations.

The syntax of a formula $\varphi$ in LTL for $K$ is $\varphi ::= \top
\:|\: p \:|\: \neg \varphi \:|\: \varphi \land \varphi \:|\: \bigcirc
\varphi \:|\: \varphi \: \mathcal{U} \: \varphi$, where $p \in P$.
Let $\cal F$ be the set of all formulas in LTL for $K$.  An arbitrary
path $\pi \in P$ of $K$ and an arbitrary LTL formula $\varphi \in \cal
F$ of $K$, $K, \pi \models \varphi$ is inductively defined as $K, \pi
\models \top$, $K, \pi \models p$ iff $p \in L(\pi(0))$, $K, \pi
\models \neg \varphi_1$ iff $K, \pi \not\models \varphi_1$, $K, \pi
\models \varphi_1 \land \varphi_2$ iff $K, \pi \models \varphi_1$ and
$K, \pi \models \varphi_2$, $K, \pi \models \bigcirc \ \varphi_1$ iff
$K, \pi^1\models \varphi1$, and $K, \pi \models \varphi_1 \:
\mathcal{U} \: \varphi_2$ iff there exists a natural number $i$ such
that $K, \pi^i\models \varphi_2$ and for all natural numbers $j < i$,
$K, \pi^j \models \varphi_1$, where $\varphi_1$ and $\varphi_2$ are
LTL formulas.  Then, $K \models \varphi$ iff $K, \pi \models \varphi$
for each computation $\pi \in C$ of $K$.  The temporal connectives
$\bigcirc$ and $\mathcal{U}$ are called the next connective and the
until connective, respectively. The other logical and temporal
connectives are defined as usual as follows: $\bot \triangleq
\neg\top$, $\varphi_1 \lor \varphi_2 \triangleq \neg(\neg\varphi_1
\land \neg\varphi_2)$, $\varphi_1 \Rightarrow \varphi_2 \triangleq
\neg\varphi_1 \lor \varphi_2$, $\lozenge \varphi \triangleq \top
\ \mathcal{U}\ \varphi$, $\square \varphi \triangleq \neg(\lozenge
\neg\varphi)$ and $\varphi_1 \leadsto \varphi_2 \triangleq \square
(\varphi_1 \Rightarrow \lozenge \varphi_2)$.  The temporal connectives
$\lozenge$, $\square$ and $\leadsto$ are called the eventually
connective, the always connective and the leads-to connective,
respectively.

%There are multiple possible ways to express states. We
%express a state as a braced associative-commutative (AC)
%collection of name-value pairs. AC collections are called
%soups, and name-value pairs are called observable components. That is, a state is expressed as a braced soup of
%observable components. 

In this paper, to express a state of $S$, we use an
associative-commutative collection of name-value
pairs. Associative-commutative collections are called soups, and
name-value pairs are called observable components. That is, a state is
expressed as a soup of observable components.  The juxtaposition
operator is used as the constructor of soups. Let $oc1, oc2, oc3$ be
observable components, and then $oc1\ oc2\ oc3$ is the soup of those
three observable components. A state is expressed as
$\{oc1\ oc2\ oc3\}$. There are multiple possible ways to specify state
transitions.  In this paper, we use Maude~\cite{Clavel2007LNCS}, a
programming/specification language based on rewriting logic, to
specify them as rewrite rules.  Maude makes it possible to specify
complex systems flexibly and is also equipped with model checking
facilities (a reachability analyzer and an LTL model checker).
%A conditional rewrite rule (or just
%a rule) is in the form \verb!crl! [$lb$] : $l => r if \ldots /\backslash c_i /\backslash \ldots$ , 
%where $lb$ is the label given to the rule and $c_i$ is part
%of the condition, which may be an equation $lc_i = rc_i$. 
%The negation of lci = rci could be written as $(lc_i$ =/= $rc_i) =$ \verb!true!, where \verb!= true! could be omitted. 
A rewrite rule starts with the keyword \verb!rl!, followed by a label enclosed with square brackets and a colon, two patterns (terms that may contain variables) connected with =\textgreater, and ends with a full stop. A conditional one starts with the keyword \verb!crl! and has a condition following the keyword \verb!if! before a full stop.
The following is a form of a conditional rewrite rule:

\smallskip
\noindent
\verb!crl! [$lb$] : $l$ =\textgreater $\ r$ \verb!if! $\ldots$ \verb!/\ !$c_i$ \verb!/\ !$\ldots$
\smallskip

\noindent
where $lb$ is a label and $c_i$ is part of the condition, which may be an equation $lc_i = rc_i$. 
The negation of $lc_i = rc_i$ could be written as $(lc_i =$\verb!/!$= rc_i) =$ \verb!true!, where \verb!= true! could be omitted. 
If the condition
$\ldots$ \verb!/\ !$c_i$ \verb!/\ !$\ldots$ holds under some substitution $\sigma$, $\sigma(l)$ can be replaced with $\sigma(r)$.

Maude provides the search command that allows finding a state reachable from
$t$ such that the state matches $p$ and satisfies condition(s) $c$:

\medskip
%\begin{small}
	\noindent
	\verb!search [n,m] in MOD! $:\ t$ =\textgreater* $p$ \verb!such that! $c$ .
%\end{small}
\medskip

\noindent
where \verb!MOD! is the name of the module specifying the state
machine, and \verb!n! and \verb!m! are optional arguments stating a
bound on the number of desired solutions and the maximum depth of the
search, respectively.  \verb!n! typically is 1 and $t$ typically
represents an initial state of the state machine.

 Let \verb!init! be the only initial state of $K$ and $\varphi$ be an LTL
 formula. Then, the Maude LTL model checker checks that
 $K$ satisfies $\varphi$ by the following command:
 
 \smallskip
 \begin{small}
 	\noindent
 	\verb!red modelCheck(init,!$\varphi$\verb!) .!
 \end{small}
 \smallskip
 
 \noindent
 where \verb!red! is an abbreviation of \verb!reduce!. 
 Executing this command, Maude will return either true if $\varphi$ is satisfied, or a counterexample when $\varphi$ is not satisfied.


 
\section{The Aoki-Rajkumar Merging Protocol}
 \label{sect_oriproto}

A merge point is visualized in Fig.~\ref{mergePoint_fig}, which is the
intersection of two lanes: one through lane (3) and one non-through
lane (1).  We suppose that each lane is one-way direction.  The
directions are shown as in Fig.~\ref{mergePoint_fig}. A vehicle
running on each lane goes though the merge point, moving to (2) in
Fig.~\ref{mergePoint_fig}. Hereinafter, let us use through lane
vehicles and non-through lane vehicles to talk about vehicles running
on the through lane and vehicles running on the non-through lane,
respectively.  Passing through the merge point, through lane vehicles
usually have a higher priority than non-through lane vehicles because
the traffic volume of the through lane is greater than that of the
non-through lane. In other words, non-through lane vehicles should let
through lane vehicles pass through the merge point first in order to
guarantee that traffic can be efficient and there is no collision.
The merge point must be controlled so that vehicles never collide with
each other.  That is to say, it is necessary to guarantee that there
is at most one vehicle passing through the merge point.


\begin{figure}[t]
\begin{center}
\scalebox{0.3}{\includegraphics{Pictures/merge_point.pdf}}
\end{center}
\caption{Merge point}
\label{mergePoint_fig}
\end{figure}

S. Aoki and R. Rajkumar~\cite{10.1145/3055004.3055028} have proposed a
merging protocol for autonomous vehicles. The protocol has two
versions that correspond to two traffic environments: (1) only
autonomous vehicles on the traffic (homogeneous traffic) and (2)
autonomous vehicles and human-driven vehicles on the traffic
(heterogeneous traffic).  In the present paper, we only concentrate on
the first version (1) that is called the Aoki-Rajkumar merging
protocol or simply the AR merging protocol.  The protocol requires
each vehicle to be equipped with software systems, such as a
navigation system and hardware devices, such as sensors.  Furthermore,
the authors also assume that each vehicle is able to uses some
technologies, such as WAVE \cite{4346439,5888501} and GPS, to
communication with other vehicles.

All vehicles participating in the protocol are controlled based on two
system modes: prioritized mode and fair mode.  When the system is in
one mode, it can change to the other mode depending on the traffic of
the through lane. Each mode is explained in the upcoming sub-sections.

\subsection{Prioritized mode}

When the system is in prioritized mode, non-through lane vehicles
basically cannot intervene through lane vehicles that are approaching
the merge point.  It means that non-through lane vehicles are allowed
to enter the merge point only if there is no through lane vehicle that
is approaching the merge point.  There are basically two cases:

\begin{itemize}
\item if there are some through lane vehicles that are approaching the
  merge point and there is no enough space between any of two adjacent
  such vehicles, non-through lane vehicles must stop before the merge
  point until all such through lane vehicles have passed through the
  merge point;

\item if there is no through-lane vehicle that is approaching the
  merge point, non-through lane vehicles are allowed to pass through
  the merge point.
\end{itemize}

\noindent
There is actually one more case:

\begin{itemize}
\item if there are some through lane vehicles that are approaching the
  merge point and there is enough space between some of two adjacent
  such vehicles, non-through lane vehicles can use the space to enter
  the merge point.
\end{itemize}

\noindent
In the AR merging protocol~\cite{10.1145/3055004.3055028}, prioritized
mode is classified into fully prioritized mode and half prioritized
mode that correspond to the first two cases and the third case,
respectively. 

% \begin{figure}[h]
% \begin{center}
% \scalebox{0.33}{\includegraphics{Pictures/fullyPrioritized.pdf}}
% \end{center}
% \caption{Fully-prioritized state}
% \label{fullyPri_fig}
% \end{figure}


%\subsection{Semi-prioritized mode}
%\label{semi-case}
%
%When the system is in semi-prioritized mode, non-through lane vehicles
%have a chance to enter the merge point even though there are some
%through lane vehicles that are approaching the merge point.  If there
%is enough space between two through lane vehicles that are approaching
%the merge point, a non-through lane vehicle that is approaching the
%merge point can use that space to enter the merge point.
%
% \begin{figure}[h]
% \begin{center}
% \scalebox{0.33}{\includegraphics{Pictures/semiPrioritizedNonthrough.pdf}}
% \end{center}
% \caption{Fully-prioritized state}
% \label{semiPriNT}
% \end{figure}

\subsection{Fair mode}

Prioritized mode can change to fair mode when the traffic of the
through lane becomes congested. In fair mode, through lane vehicles
and non-though lane vehicles are allowed to enter the merge point
alternately. Fair mode can change back to prioritized mode when the
traffic of the through lane becomes less congested.

\subsection{Part of the pseudo-code}

Due to the space limitation, we cannot describe the AR merging
protocol in pesudo-code in detail. Let us show part of the pseudo-code:

\begin{small}
\begin{tabbing}
\=\ \ \ \ \=\ \ \=\kill
\>\>\textbf{else if} $\textrm{RM}(\omega_{l1}) = \textrm{APPROVE}$ \textbf{then} \\
\>\>\>Start the maneuver; \\
\>\>\textbf{else if} $\textrm{Average}\{\upsilon_{\omega_{h1}},\ldots,\upsilon_{\omega_{hn}}\}  < \psi$ \textbf{then} \\
\>\>\>$\textrm{SM}(\omega_{h1}) = \textrm{ENTER}$; \\
\>\>\>\ldots
\end{tabbing}
\end{small}

\noindent
This says the following: if the top non-thourgh lane vehicle
$\omega_{l1}$ receives the message $\textrm{APPROVE}$ from a through
lane vehicle, then $\omega_{l1}$ enters the merge point; if the
average speed of the multiple thorugh lane vehicles
$\upsilon_{\omega_{h1}},\ldots,\upsilon_{\omega_{hn}}$ is less than
$\psi$, $\omega_{l1}$ sends the message $\textrm{Average}$ to the top
through lane vehicle $\omega_{h1}$.

% \begin{figure}[t]
% \begin{center}
% \scalebox{0.33}{\includegraphics{Pictures/fairState.pdf}}
% \end{center}
% \caption{Fair state}
% \label{semiPriNT}
% \end{figure}




% Algorithm 1, Algorithm 2 and Algorithm 3 present the algorithm of \textit{Autonomous Vehicle Protocol for Merge Points}. Algorithm 1 presents the protocol for the vehicle on the non-through lane, and Algorithm 2 presents the protocol for the vehicle on the through lane. In addition, Algorithm 3 presents the protocol for the vehicle on the non-through lane when the vehicle is in Fair state. The vehicles on the non-through lane, originally having the lower priority which are represented by $\omega_{l1}$, .. $\omega_{ln}$. The vehicles on the through lane originally having higher priority are denoted as $\omega_{h1}$, ... $\omega_{hn}$. The sequential subscript numbers indicate the vehicle order, meaning $\omega_{l1}$ is the leader vehicle on the non-through lane before the merge intersection.



% The CROSS message is not used in any algorithms because this perception is sufficiently safe for the intersection. However, it is used in order to enhance the reliability and safety of the protocol.


% \begin{algorithm}[t]
% \caption{Protocol for Vehicle $\omega_{l1}$ on non-through lane} 
% \label{alg:nlane}
% \While{\textit{not started the maneuver}}
% {
% SM( $\omega_{l1}$) = \textbf{REQUEST};\newline
% \uIf{\textit{not halting before the merge point}}
% {
% \uIf{\textit{No vehicle before the merge point}}{Start the maneuver;}
% \uElseIf{\textup{RM($\omega_{l1}$) = \textbf{APPROVE}}}{Start the maneuver;}
% \Else{Slow down (for halting right before the merge point);}
% }
% \Else{
% Halting before the merge point($\upsilon_{\omega_{h1}}$ = 0);\newline
% \uIf{\textit{not halting before the merge point}}{Start the maneuver;}
% \uElseIf{\textup{RM($\omega_{l1}$) = \textbf{APPROVE}}}{Start the maneuver;}
% \uElseIf{Average\{$\upsilon_{\omega_{h1}}$,..$\upsilon_{\omega_{hn}}$\}  $<$ $\psi$}
% {SM($\omega_{h1}$) = \textbf{ENTER};\newline
% SM($\omega_{l2}$, ..$\omega_{ln}$) = \textbf{STATE TRANSITION};\newline
% $\omega_{h1}$  transits to \textit{Fair state};\newline
% Start the maneuver;
% }
% \Else{
% \textit{k} = arg max \textit{D\{$\omega_{hi}$\}};\newline
% SM($\omega_{hk}$) = \textbf{INTERRUPT};\newline
% \If{\textup{RM($\omega_{hk}$) = \textbf{YIELD}}}
% {
% \While{$s_{1}$ sec}
% {
% \If{\textup{RM($\omega_{h(k-1)}$ = \textbf{EXIT}}}{Start the maneuver;}
% }
% }
% }
% }
% }
% \end{algorithm}


% Some other notations used in these algorithms.
% \begin{itemize}
%     \item $\upsilon_{\omega_{hk}}$: Current speed of the vehicle $\omega_{hk}$
%     \item $\psi$: Speed threshold
%     \item $s_{1}$: Time threshold for packet loss (for \textbf{INTERRUPT} or \textbf{YIELD})
%     \item $s_{2}$: Time threshold for packet loss (for \textbf{EXIT)}
%     \item $\Delta$: Time threshold for \textbf{ABOUT TO EXIT}
% \end{itemize}


% \begin{algorithm}[t]
% \caption{Protocol for Vehicle $\omega_{h1}$ on through lane}
% \label{alg:tlane}
% \uIf{RM($\omega_{h1}$) = \textbf{\textup{REQUEST}}}
% {
% \uIf{CollisionDetect($\omega_{h1}$, $\omega_{l1}$) = true}
% {SM($\omega_{l1}$) = \textbf{DECLINE};\newline Start the maneuver;}
% \Else{SM($\omega_{l1}$) = \textbf{APPROVE};\newline Start the maneuver;}
% }
% \uElseIf{RM($\omega_{h1}$) = \textbf{\textup{ENTER}}}
% {
% \While{$s_2$sec}
% {
% \uIf{RM($\omega_{l1}$) = \textbf{\textup{ABOUT TO EXIT}}}{
% SM($\omega_{l1}$) = \textbf{ENTER};\newline
% Going to start the maneuver;\newline
% \uIf{RM($\omega_{l1}$) $\neq$ \textbf{\textup{EXIT}} in $\Delta$sec later after \textbf{\textup{ABOUT TO EXIT}}}{Slow down and Halting;}
% \Else{Start the maneuver;}
% }
% \uElseIf{RM($\omega_{l1}$) = \textbf{\textup{EXIT}}}{Start the maneuver;}
% \Else{Halting;}
% }
% \uIf{No vehicle before the merge point}{Start the maneuver;}
% \Else{Halting;}
% }
% \uElseIf{RM($\omega_{l1}$) = \textbf{\textup{INTERRUPT}}}{\uIf{CollisionDetect($\omega_{h1}$, $\omega_{l1}$) = true}{Slow down(to make space for $\omega_{l1}$);\newline
% SM($\omega_{l1}$) = \textbf{YIELD};}
% \Else{SM($\omega_{l1}$) = \textbf{YIELD};}}
% \Else{Start the maneuver;}
% \end{algorithm}

% \begin{algorithm}[t]
% \caption{Protocol for Vehicle $\omega_{l1}$ in Fair state}
% \label{alg:fair}
% \uIf{right before the merge point}{
% \uIf{RM($\omega_{h1}$) = \textbf{\textup{ABOUT TO EXIT}}}{
% SM($\omega_{h2}$) = \textbf{ENTER};\newline
% SM($\omega_{l2}$, ..$\omega_{ln}$) = \textbf{STATE TRANSITION};\newline
% Going to start the maneuver;\newline
% \uIf{RM($\omega_{h1}$) $\neq$ \textbf{\textup{EXIT}} in $\Delta$sec later after \textbf{\textup{ABOUT TO EXIT}}}{Halting;}
% \Else{Start the maneuver;}
% }
% \uElseIf{RM($\omega_{h1}$) = \textbf{\textup{EXIT}}}{
% SM($\omega_{h2}$) = \textbf{ENTER};\newline
% SM($\omega_{l2}$, ..$\omega_{ln}$) = \textbf{STATE TRANSITION};\newline
% Start the maneuver;
% }
% \Else{Halting;}
% }
% \Else{Follow the vehicle in front;}
% \end{algorithm}

% In the \textit{Autonomous Vehicle Protocol for Merge Points}, three functions implies three priority state of non-through lane vehicles. (i)High throughput function; (ii) Intervention function; (iii) Zipper merge function. 


\section{A Revised Version of the Protocol}
 \label{sect_reviproto}

We revise the AR merging protocol such that it does not rely any
real-time information, such as the average speed of multiple vehicles
running on a lane. Or we can say that we design a new merging protocol
that does not rely on such information based on the AR merging
protocol. Because the speed of a vehicle may drastically change in a
short moment, say by sudden breaking, we do not think that the average
speed of multiple vehicles is reliable enough to safely control
autonomous vehicles. The revised version of the AR merging protocol is
called our merging protocol (or our protocol) in the present paper.
Each vehicle in our protocol is attached to one of six statuses:
\textit{running}, \textit{approaching}, \textit{stopped},
\textit{crossing}, \textit{crossed}, \textit{halting}.  When a vehicle
is far away from the merge point, its status is running. When a
vehicle gets close to the merge point, its status changes to
approaching from running. When a vehicle stops before the merge point,
its status is either stopped or halting. The difference between
stopped and halting will be described later. When a vehicle enters the
merge point, its status changes to crossing from approaching, stopped
or halting. When a vehicle has passed through the merge point, its
status changes to crossed from crossing. We suppose that vehicles
whose statuses are approaching, stopped or halting on one lane never
pass over any other vehicles. To this end, we need to use V2V
communications. Such vehicles can be regarded as making a virtual
queue of vehicles. As the AR merging proocol, there are two modes
(priotitized mode and fair mode) in our merging protocol. Initially,
we suppose that the system is in prioritized mode and each vehicle is
in running.

In our protocol, there are three cases in which a non-through lane
vehicle is allowed to enter the merge point: (i) no through lane
vehicle is approaching the merge point; (ii) some through lane
vehicles are approaching the merge point such that there is enough
space between some two adjacent vehicles; (iii) through lane and
non-through lane vehicles enter the merge point alternately when the
traffic of the through lane is congested.

\begin{figure}[h]
\begin{center}
\scalebox{0.3}{\includegraphics{Pictures/throughLane.pdf}}
\end{center}
\caption{Change of the status of a through lane vehicle}
\label{throughLaneStatus}
\end{figure}

Fig.~\ref{throughLaneStatus} visualizes the change of the status
of a through lane vehicle.  Each arrow represents the change of the
vehicle status from one value to another if the condition (or the
trigger event) denoted by the number attached to the arrow is
satisfied.  Each condition (or trigger event) is:
\begin{enumerate}[]
    \item The vehicle gets close to the merge point. %1
    \item The system is in prioritized mode and there is no vehicle crossing the merge point. %2
    \item The vehicle has crossed through the merge point. %3
    \item The system is in fair mode, it is the through lane vehicle's turn and there is no vehicle crossing the merge point. %5
    \item The  system is in fair mode and it is the non-through lane vehicle's turn. %6
    \item The  system becomes prioritized mode. %7
    \item The system is in fair mode, it is the through lane vehicle's turn and there is no vehicle crossing the merge point. %8
\end{enumerate}


%    \item The system is in prioritized mode, the through lane is congested (i.e., the number of the vehicles in the virtual queue on the through lane is greater than a specific number) and there is no vehicle crossing the merge point.  %4

When a through lane vehicle becomes approaching from running, the
system is in prioritized mode and the number of the through lane
vehicles that constitute the virtual queue on the through lane is
greater then a specific number, the system becomes fair mode. Note that
the through lane vehicles that constitute the virtual queue on the
through lane are in approaching or halting. Note also that the traffic
of the through lane is congested if and only if the number of the
through lane vehicles that constitute the virtual queue on the through
lane is greater then a specific number.  When a through lane vehicle
becomes crossing and the system is in fair mode, the system becomes
prioritized mode if the number of the through lane vehicles that
constitute the virtual queue on the through lane becomes less than or equal
to a specific number.

Fig.~\ref{space} visualizes how a non-through lane vehicles can enter
the merge point during a semi-prioritized state when there is a space
between two vehicles running on the through lane.  The blue circle
represents the non-through lane vehicle, the red circle represents the
through lane vehicles, and the white circle represents a space.  Green
area denotes the merge point.  Each vehicle or space is attached a
different number.
%Figure \ref{space}.(a) shows that vehicle 1 on the non-through lane is approaching the merge point.
In the Fig.~\ref{space} (a), there are two through lane vehicles (i.e., \textcircled{2} and \textcircled{4}), and there is a space (i.e., \textcircled{3}) in the middle of the two vehicles that is enough to accommodate the third vehicle. 
The vehicles \textcircled{2} and \textcircled{4} are approaching the intersection.
According to the full-prioritized state, the vehicle \textcircled{1} needs to let the vehicles \textcircled{2} and \textcircled{4} cross the merge point first. 
Thus, as shown in Fig.~\ref{space} (b), the vehicle \textcircled{1} stops right before the merge point, while the vehicle \textcircled{2} passes through the intersection.
In Fig.~\ref{space} (c) and Fig.~\ref{space} (d), the vehicle \textcircled{1} enters the merge point because there exists a space on the through lane between \textcircled{2} and \textcircled{4}.
%Fig.~\ref{space} (c) and Fig.~\ref{space} (d) show the process of the vehicles on the non-through lane find that it has enough space to join the vehicles on the through lane and enter the merge point.

\begin{figure}[h]
	\begin{center}
		\scalebox{0.28}{\includegraphics{Pictures/space.pdf}}
	\end{center}
	\caption{An example of semi-prioritized state}
	\label{space}
\end{figure}

\begin{figure}[h]
\begin{center}
\scalebox{0.36}{\includegraphics{Pictures/nonThroughLane.pdf}}
\end{center}
\caption{Transition of a non-through lane vehicle's status}
\label{nonThroughLaneStatus}
\end{figure}

Fig.~\ref{nonThroughLaneStatus} shows the transition of a non-through lane vehicle's status.
Transitions are explained as follows:
\begin{itemize}
    \item[a)] the vehicle is approaching the merge point. %1
    \item[b)] there exists a through lane vehicle that is approaching the merge point, and the system is not in a fair state. %2
    \item[c)] there is no vehicle on the through lane that is approaching the merge point, and the system is not in a fair state, and there is no vehicle crossing the merge point (from \textit{approaching} to \textit{crossing}). %3
    \item[d)] the vehicle crossed the merge point. %4
    \item[e)] there is no vehicle on the through lane that is approaching the merge point, and the system is not in a fair state, and there is no vehicle crossing the merge point (from \textit{stopped} to \textit{crossing}). %5
    \item[f)] the system is in a semi-prioritized state, and there is no vehicle crossing the merge point. %6
    \item[g)] the system is in a fair state, and it is the non-through lane vehicle's turn, and there is no vehicle crossing the merge point. %7
    \item[h)] the system is in a fair state, and it is the through lane vehicle's turn (from \textit{approaching} to \textit{stoped}). %8
    \item[i)] the system is not in a fair state. %9
    \item[j)] the system is in a fair state, and it is the through lane vehicle's turn (from \textit{stopped} to \textit{halting}). %10
    \item[k)] the system is in a fair state, and it is the non-through lane vehicle's turn, and there is no vehicle crossing the merge point (from \textit{halting} to \textit{crossing}). %11
    \item[l)] the system is in a fair state, and it is the non-through lane vehicle's turn, and there is no vehicle crossing the merge point (from \textit{stopped} to \textit{crossing}). %12
\end{itemize}

Initially, all non-through lane vehicles receive \textit{running} as their statuses.
It changes to \textit{approaching} status when it approaches the merge point.
Let us consider the case when the system is in a fully-prioritized state first.
The non-through lane vehicle initially has lower priority than other vehicles running on the through lane.
Therefore, its status changes to \textit{stopped} from \textit{approaching} when there exists another through lane vehicle approaching to the merge point.
If there does not exist a vehicle running on the through lane with \textit{approaching} status, the vehicle changes its status to \textit{crossing} from \textit{stopped}.

The second case is when the system is in a fully-prioritized state.
If there exists a space between the merge point and another through lane vehicle whose status is \textit{approaching}, 
the vehicle changes its status to \textit{crossing} from \textit{stopped}.
%If vehicle on non-through lane finds no vehicle on through lane is approaching to the merge point, it will go to \textit{crossing} status from \textit{stopped} status. 
%If vehicle on non thorough lane find space before the merge point it knows Semi-prioritized state comes and it changes to \textit{crossing} status from \textit{stopped} status. After vehicle crossed the merge point, it moves to \textit{crossed} status.

When the system in a fair state, which is the last case.
%While vehicle on through lane changes to the Fair state, vehicle on non-through lane also moves to Fair state. 
If it is the through lane vehicle's turn, the vehicle's status changes to \textit{halting} from \textit{approaching} or \textit{stopped}. 
On the other hand, it changes the status to \textit{crossing} from \textit{approaching}, \textit{stopped} or \textit{halting}. 
 


 
\section{Formal Specification}
 \label{sect_formal}
%In the semi-prioritized state of the merging protocol, if there exists an enough space on through lane, a non-through lane vehicle can enter to the merge point.
We use two queues to maintain the vehicles running on the through lane and non-through lane, respectively.
Each element of the queue can be not only a vehicle, but also a space.
 %From the \textit{Revised Autonomous Vehicle Protocol for Merge Points }, we know that in the Semi-priority state, a space can accommodate a car, so we can divide the road into many small squares, and each small square can accommodate a car or a space or nothing.
%Furthermore, we use a global variable that contains 5 
%  (\verb!v![$vid$] : $lid$\verb!,!$vstat$\verb!,!$t$\verb!,!$lt$)
In this paper, a state is expressed as a soup of observable components.
To formalize the protocol, we use four observable components as follows:
\begin{itemize}
    \item (\verb!v[!$id$\verb!]:!$position$\verb!,!$vstat$) - $id$ is an ID (a nature number) of a vehicle or a space, $position$ is the lane that the vehicle or the space $id$ is located (either through lane or non-through lane), $vstate$ is the status of the vehicle if it is a vehicle. Initially, $vstate$ is \verb!space! if it is a space, or \verb!running! if it is a vehicle.
    \item (\verb!lane[!$position$\verb!]:!$q$) - $position$ is the through lane or non-through lane, and $q$ is a queue of vehicles and spaces. Initially, $q$ is \verb!empq!, which denotes an empty queue.
    
	\item (\verb!vehicleCrossing:!$b1$) - $b1$ is either true or false, indicating that there exists a vehicle crossing the merge point or there does not, respectively. Initially, $b1$ is false.
	
	\item (\verb!mode:!$smode$) - $smode$ receives one of the following four values: 
	\verb!fullP! indicating that the system is in fully-prioritized mode, 
	\verb!semiP! indicating that the system is in semi-prioritized mode, 
	\verb!fairT! indicating that the system is in fair mode and it is the through lane vehicle's turn, or 
	\verb!fairN! indicating that the system is in fair state and it is the non-through lane vehicle's turn.
	Initially, $smode$ is \verb!fullP!.
	
	\item (\verb!veInNTLane:!$b2$) - $b2$ is either true or false, indicating that there exists a non-through lane vehicle approaching the merge point or there does not, respectively. Initially, $b1$ is false.
	
	\item (\verb!numVeInTLane:!$count$) - $count$ is the total number of through lane vehicles approaching the merge point. Initially, $count$ is set to 0.

    \item (\verb!gstat! : $gstat$) -  $gstat$ is either \verb!fin! or \verb!nFin!.
	When it is \verb!fin!, all vehicles concerned have crossed the intersection.
\end{itemize}

We use 16 rewrite rules to define the transitions of the protocol.
%Let us consider a case such that the vehicle approaches the merge point, the status of the vehicle changes from the \verb!running! status to the \verb!approaching! status. 
%This case corresponds to condition 1 of two lanes.
%We use four rules to specify this case in which
%We use four rules to specify this process. 
%two rules for the through lane and two rules for the non-through lane.
The following two rules are defined to specify a set of transitions that
change a vehicle status from \verb!running! to \verb!approaching!, one for through lane vehicles, and another for non-through lane vehicles: 

\begin{small}
\begin{verbatim}
crl [approach_T] : {(gstat: nFin) (mode: M) 
(veInNTLane: B3) (numVeInTLane: NT) 
(lane[L]: VS) (v[VI]: L,running) OCs} 
=> {(gstat: nFin) (mode: 
(if NT >= 2 and B3 then fairT else M fi))
(veInNTLane: B3) (numVeInTLane: (NT + 1)) 
(lane[L]: (VS ; VI)) (v[VI]: L,approaching) 
OCs} 
if L == Through .
\end{verbatim}
\end{small}
\begin{small}
\begin{verbatim}
crl [approach_N] : {(gstat: nFin) 
(veInNTLane: B3) 
(lane[L]: VS) (v[VI]: L,running) OCs} 
=> {(gstat: nFin) (veInNTLane: true) 
(lane[L]: (VS ; VI)) (v[VI]: L,approaching) 
OCs} 
if L == Nothrough .
\end{verbatim}
\end{small}

\noindent 
where \verb!B3! is a Maude variable of Boolean values,
\verb!M! is a Maude variable recieving one of these values: \verb!fullP!, \verb!semiP!, \verb!fairT!, or \verb!fairN!.
\verb!NT! is a Maude variable of natural numbers.
\verb!VI! is a Maude variable of natural numbers, denoting arbitrary vehicle IDs. 
\verb!VS! is a Maude variable of queues, possibly empty.
\_;\_ is the constructor of queues, where an underscore \_ is
a place holder where an argument (e.g., vehicle ID) is put. 
The first rewrite rule says that when the through lane has a vehicle whose status is \verb!running!, the status is changed to \verb!approaching!, the vehicle is enqueue into the queue of the through lane, 
the value of the observable component \verb!numVeInTLane! is increment,
and the system change to semi-prioritized mode if that value is greater or equals to 3.
%It means that \textit{(q1 ; q2) ; q3 = q1 ; (q2 ; q3)}. 
%\verb!empq! represents the queue is empty. 
%\verb!(numOfVeInTLane)! in the \verb!switch! means the length of queue on through lane (only count vehicle).
%%From \verb![approach_T1]! and \verb![approach_T2]! rules, \verb!(numOfVeInTLane + 1)! in the \verb!switch! means the size of queue which represent vehicle on through lane add 1. 
%\verb!VeInNTLane! in the \verb!switch! means the queue of the non-through lane is not empty. 
%Notice, vehicles whose status are \verb!running! cannot be counted in the queue (both lanes).
%
%We use the following states to express conditions of both lanes mentioned in~\ref{sect_reviproto} as follows:


The remaining rewrite rules are divided into four groups: a vehicle tries to enter the merge point when the system is in fully-prioritized mode; a vehicle tries to enter the merge point when the system is in semi-prioritized mode; a vehicle leaves the merge point when the system is in prioritized mode; and
when the system is in fair mode.
	
\begin{figure*}[tb]
\begin{center}
\scalebox{0.35}{\includegraphics{Pictures/fullySim.pdf}}
\end{center}
\caption{An example of fully-prioritized state}
\label{fully_fig}
\end{figure*}

\subsection{A vehicle tries to enter the merge point when the system is in fully-prioritized mode}

Let us consider a concrete fully-prioritized state in which there are two through lane vehicles and one vehicle running on the non-through lane as shown in Fig.~\ref{fully_fig}. 
The initial state shown in Fig.~\ref{fully_fig} (a) is expressed as follows:
%Initially, the through lane and non-through lane both have one vehicle running on them. Here, the initial state is expressed as follows:
 
\begin{small}
\begin{verbatim}
{(gstat: nFin) (lane[Through]: empq) 
(lane[Nothrough]: empq) 
(v[1]: Through,running) 
(v[2]: Through,running)
(v[3]: Nothrough,running)
(vehicleCrossing: false) (mode: fullP) 
(veInNTLane: false) (numVeInTLane: 0)} .
\end{verbatim}
\end{small}

%\noindent \verb!v[1]! and \verb!v[2]! observable components represent two vehicles. 
%One of them is running on the through lane, the other one is running on the non-through lane. The observable component of \verb!switch! means no vehicle is crossing the merge point and the whole self-driving system is either in Semi-prioritized state or Fair state. \verb!noVeInNTLane! and \verb!0! show the queue of two lanes are empty.

Two rewrite rules are introduced in this case.
The first one \verb!full_T_crossing! specifies the set of transitions 2) explained in the previous section.
That is a through lane vehicle is allowed to pass the merge point if there is no vehicle crossing the merge point and the system is in a fully-prioritized state.
Fig.~\ref{fully_fig} (b) and (c) graphically visualize this kind of transition.
The rewrite rule is defined as follows:
%Condition 2 of the through lane is shown by rule \verb![full_T_crossing]!:

\begin{small}
\begin{verbatim}
crl [full_T_crossing] : {(gstat: nFin) 
(vehicleCrossing: false) (mode: fullP) 
(lane[L]: (VI ; VS)) (v[VI]: L,VeSt) OCs} 
=> {(gstat: nFin) (vehicleCrossing: true) 
(mode: fullP) (lane[L]: (VI ; VS)) 
(v[VI]: L,crossing) OCs} 
if L == Through /\ 
(VeSt == approaching or VeSt == stopped) .
\end{verbatim}
\end{small}

\noindent
where \verb!VeSt! is Maude variable of sort \verb!VStat!, denoting arbitrary vehicle statuses.

The second rewrite rule \verb!full_N_crossing! specifies the set of transitions c) and e) explained in the previous section.
That is a non-through lane vehicle whose status is \verb!approaching! or \verb!stopped! is allowed to enter the merge point
if there is no vehicle on the through lane that is approaching the merge point, and the system is not in a fair state.
Fig.~\ref{fully_fig} (d) and (e) visualize this kind of transition.
The rewrite rule is defined as follows:

\begin{small}
\begin{verbatim}
crl [full_N_crossing] : {(gstat: nFin)
(vehicleCrossing: false) (mode: fullP) 
(numVeInTLane: 0) (lane[L]: (VI ; VS))
(v[VI]: L,VeSt) OCs} 
=> {(gstat: nFin) (vehicleCrossing: true) 
(mode: fullP) (numVeInTLane: 0)
(lane[L]: (VI ; VS)) (v[VI]: L,crossing) OCs}
if L == Nothrough /\ 
(VeSt == approaching or VeSt == stopped) .
\end{verbatim}
\end{small}



\subsection{A vehicle tries to enter the merge point when the system is in semi-prioritized mode}

%We proposed that observable component of (\verb!v[!$id$\verb!]: !$position$\verb!,!$vstat$) is a vehicle or space. 
%As mentioned in section~\ref{semi-case}, when the system is in a semi-prioritized state, a space appears in through lane.
%If it is a space on the non-through lane, we delete it because non-through lane initially has lower priority than the through lane.
%We use the following rule to specify this situation in Maude.
%
%\begin{footnotesize}
%\begin{verbatim}
%crl [semi_N_space2] : 
%{(gstat: nFin) 
%(switch: B1,B2,B3,B4,B5) 
%(lane[L]: (VI ; VS)) (v[VI']: L,space) OCs} 
%=> {(gstat: nFin) 
%(switch: B1,B2,B3,B4,B5) 
%(lane[L]: (VI ; VS)) OCs} 
%if L == Nothrough .
%\end{verbatim}
%\end{footnotesize}

%\noindent On the other side, If it is the space on the through lane, we push the space into the queue of the through lane.
%When the system is in a semi-prioritized state, and there exists a space on the through lane, we push the space into the queue of the through lane.
We introduce the following rewrite rule:
\begin{small}
\begin{verbatim}
crl [semi] : {(gstat: nFin) 
(vehicleCrossing: false) (mode: M) 
(veInNTLane: true) (lane[Through]: (VI ; VS)) 
(v[VI]: Through,space) 
(lane[Nothrough]: (VI' ; VS'))
(v[VI']: Nothrough,approaching) OCs} 
=> {(gstat: nFin) (vehicleCrossing: true) 
(mode: semiP) (veInNTLane: true)
(lane[Through]: VS) (v[VI]: Through,yield) 
(lane[Nothrough]: (VI' ; VS'))
(v[VI']: Nothrough,crossing) OCs}
if (M == fullP or M == semiP) .
\end{verbatim}
\end{small}

\noindent
where \verb!VI'! and \verb!VS'! are Maude variable of natural numbers and queues, respectively.
The rewrite rule says that when there is no vehicle crossing the merge point, the system in fully/semi-prioritized mode, 
and a space is the top element of the through lane's queue, then the top vehicle of the non-through lane's queue, which is approaching the merge point, is allowed to enter the merge point, the system changes to semi-prioritized mode, and the space is not available anymore.

\subsection{A vehicle leaves the merge point when the system is in prioritized mode}
%Then, let us consider vehicles' leaving of two lanes in fully-prioritized state and semi-prioritized state. 
%Firstly, no matter which vehicle crossed, the whole system should return to fully-prioritized state. 
%Firstly, the whole system returns to fully-prioritized if a vehicle has just crossed to the merge point.
%Secondly, we remove one vehicle out of the queue when it crossed and update the number of vehicle on the through lane.
%In non-through lane, \verb!veInNTLane! is changed to \verb!noVeInNTLane! if the last vehicle on non-through lane crossed the merge point.
%The followings rules refer to some cases of vehicle leaving on through lane and non-through lane.
%Secondly, we remove one vehicle out of the queue when it crossed, therefore, number of vehicle on the through lane should minus 1 and \verb!switch! should tell us whether ti is the last element in the queue of the non-through lane.
%Basing on these considerations, we specify vehicle leaving on through lane and non-through lane as follows:


%When a vehicle leaving the merge point, it is unqueue from the corresponding queue.

The following rewrite rule specifies a set of transitions when a through lane vehicle leaving the merge point:

\begin{small}
\begin{verbatim}
crl [T_leave] : {(gstat: nFin) 
(vehicleCrossing: true) (mode: M) 
(numVeInTLane: NT) 
(lane[L]: (VI ; VS)) (v[VI]: L,crossing) OCs} 
=> {(gstat: nFin) (vehicleCrossing: false) 
(mode: (if (VS == empq or NT <= 3) then 
fullP else M fi)) (numVeInTLane: sd(NT,1)) 
(lane[L]: VS) (v[VI]: L,crossed) OCs} 
if L == Through /\ 
(M == fullP or M == semiP) .
\end{verbatim}
\end{small}

\noindent
The rewrite rule says that in the prioritized mode, when a through lane vehicle leaving the merge point, it is deleted from the queue of the through lane, the value of observable component \verb!vehicleCrossing! is set to false, 
the value of observable component \verb!numVeInTLane! is decrement, 
and the system mode is set to fully-prioritized if the queue becomes empty or the number of through lane vehicles approaching the merge point is smaller than 3.

The following rewrite rule specifies a set of transitions when a non-through lane vehicle leaving the merge point:

\begin{small}
\begin{verbatim}
crl [N_leave] : {(gstat: nFin) 
(vehicleCrossing: true) (mode: M) 
(veInNTLane: true) (lane[L]: (VI ; VS)) 
(v[VI]: L,crossing) OCs} 
=> {(gstat: nFin) (vehicleCrossing: false) 
(mode: fullP) (veInNTLane: 
(if VS == empq then false else true fi)) 
(lane[L]: VS) (v[VI]: Through,crossed) OCs} 
if L == Nothrough /\ 
(M == fullP or M == semiP) . 
\end{verbatim}
\end{small}

\noindent
The rewrite rule says that in the prioritized mode, when a non-through lane vehicle leaving the merge point, it is deleted from the queue of the non-through lane, the value of observable component \verb!vehicleCrossing! is set to false, 
the system mode changes to fully-prioritized, and 
the value of observable component \verb!veInNTLane! is set to false (i.e., there is not any vehicle on the non-through lane) if the queue becomes empty.

%\noindent Rule \verb![T_leave1]! shows vehicle on through lane leaving in fully-prioritized state and semi-prioritized state while rule \verb![N_leave2]! shows non-through lane.
%\noindent Rule \verb![T_leave1]! can be applied for the vehicle on through lane leaving in fully-prioritized state and semi-prioritized state; while rule \verb![N_leave2]! for the vehicle on non-through lane as shown in Fig.~\ref{fully_fig} (e) and (f).


\subsection{The system is in fair mode}
When the system is in fair mode and it is the non-through
lane vehicle's turn, the top through lane vehicle must be stop right before the merge point.
The rewrite rule \verb!fair_T_stop! is introduced as follows:

\begin{small}
	\begin{verbatim}
crl [fair_T_stop] : {(gstat: nFin) 
(mode: fairN) (lane[L]: (VI ; VS)) 
(v[VI]: L,approaching) OCs} 
=> {(gstat: nFin) (mode: fairN) 
(lane[L]: (VI ; VS)) (v[VI]: L,stopped) OCs} 
if L == Through .
\end{verbatim}
\end{small}

Likewise, the rewrite rule \verb!fair_N_stop! is introduced:
\begin{small}
	\begin{verbatim}
crl [fair_N_stop] : {(gstat: nFin) 
(mode: fairT) (lane[L]: (VI ; VS)) 
(v[VI]: L,approaching) OCs} 
=> {(gstat: nFin) (mode: fairT) 
(lane[L]: (VI ; VS)) (v[VI]: L,stopped) OCs} 
if L == Nothrough .
\end{verbatim}
\end{small}

The rewrite rule \verb!fair_T_crossing! specifies the set of transitions 5) and 8) explained in the previous section.
That is when the system is in a fair state, it is the through lane vehicle's turn, and the status of top of the though lane queue is \verb!stopped! or \verb!approaching!, its status will be changed to \verb!crossing!, meaning that it is allowed to cross the merge point.
The rewrite rule is defined as follows:

\begin{small}
\begin{verbatim}
crl [fair_T_crossing] : {(gstat: nFin) 
(vehicleCrossing: false) (mode: fairT) 
(lane[L]: (VI ; VS)) (v[VI]: L,VeSt) OCs} 
=> {(gstat: nFin) (vehicleCrossing: true) 
(mode: fairT) (lane[L]: (VI ; VS))
(v[VI]: L,crossing) OCs} 
if L == Through /\ 
(VeSt == approaching or VeSt == stopped) .
\end{verbatim}
\end{small}

The rewrite rule \verb!fair_N_crossing! specifies the set of transitions g), k), and l) explained in the previous section.
That is when the system is in a fair state, it is the non-through lane vehicle's turn, and the status of top of the non-though lane queue is \verb!approaching! or \verb!stopped!, its status will be changed to \verb!crossing!, meaning that it is allowed to cross the merge point.
The rewrite rule is defined as follows:

\begin{small}
	\begin{verbatim}
crl [fair_N_crossing] : {(gstat: nFin) 
(vehicleCrossing: false) (mode: fairN) 
(veInNTLane: true) (lane[L]: (VI ; VS)) 
(v[VI]: L,VeSt) OCs} 
=> {(gstat: nFin) (vehicleCrossing: true) 
(mode: fairN) (veInNTLane: true) 
(lane[L]: (VI ; VS)) (v[VI]: L,crossing) OCs} 
if L == Nothrough /\ 
(VeSt == approaching or VeSt == stopped) .
\end{verbatim}
\end{small}

When a vehicle leaves the merge point, we not only need to remove it out of the queue, but also need to hand over the turn to the other lane, if it is not the last vehicle in queue.
On the other hand, if it is the last vehicle, we change the system back to a fully-priority state.
Two rewrite rules are introduced as follows:

\begin{small}
\begin{verbatim}
crl [fair_T_crossed] : {(gstat: nFin) 
(vehicleCrossing: true) (mode: fairT)
(veInNTLane: B3) (numVeInTLane: NT) 
(lane[L]: (VI ; VS)) (v[VI]: L,crossing) OCs} 
=> {(gstat: nFin) (vehicleCrossing: false)
(mode: (if (VS == empq or B3 == false) then 
fullP else fairN fi)) 
(veInNTLane: B3) (numVeInTLane: sd(NT,1)) 
(lane[L]: VS) (v[VI]: L,crossed) OCs} 
if L == Through .
\end{verbatim}
\end{small}

\begin{small}
	\begin{verbatim}
crl [fair_N_crossed] : {(gstat: nFin) 
(vehicleCrossing: true) (mode: fairN) 
(veInNTLane: B3) (lane[L]: (VI ; VS)) 
(v[VI]: L,crossing) OCs} 
=> {(gstat: nFin) (vehicleCrossing: false)
(mode: (if VS == empq then fullP else fairT 
fi)) 
(veInNTLane: (if VS == empq then false else
B3 fi))
(lane[L]: VS) (v[VI]: L,crossed) OCs}
if L == Nothrough .
\end{verbatim}
\end{small}


\noindent
In the rewrite rule \verb!fair_T_crossed!, before handing over the turn to the non-though lane, we need to check if there is at least one vehicle on the through lane. If there is not any such a vehicle, the system is changed to fully-prioritized mode.


There are also some more rewrite rules that are not presented here.
All of them can be found in the webpage shown in the Sect.~\ref{sect_intro}.

%In Fair state, the traffic on the through lane is congested. 
%Therefore, we use \verb!switch! to protect space from approaching the merge point in our formal specification.
%However, there is a situation that space has already in the queue of the through lane when it in Fair state. 
%We delete the space when it comes to that situation to keep the order correctly. 
%If it is the last element in the queue of the through lane, we move the state back to the Fully-prioritized state.
%
%\begin{footnotesize}
%\begin{verbatim}
%crl [fair_space1] :
%{(gstat: nFin) (switch: B1,B2,
%fairT,B4,numOfVeInTLane) 
%(lane[L]: (VI ; VI' ; VS)) (v[VI]: L,yield) OCs} 
%=> {(gstat: nFin) (switch: B1,B2
%,fairT,B4,numOfVeInTLane) (lane[L]: VI' ; VS) OCs} 
%if L == Through .
%
%crl [fair_space2] : 
%{(gstat: nFin) (switch: B1,B2,
%fairT,B4,numOfVeInTLane) 
%(lane[L]: VI) (v[VI]: L,yield) OCs} 
%=> {(gstat: nFin) (switch: B1,B2,
%noFair,B4,numOfVeInTLane) (lane[L]: empq) OCs} 
%if L == Through .
%\end{verbatim}
%\end{footnotesize}
%
%\noindent Condition 7 in \textit{Revised Autonomous Vehicle Protocol for Merge Points} of through lane and condition 9 in \textit{Revised Autonomous Vehicle Protocol for Merge Points} of non-through lane mean vehicle in halting status should back to approaching status because halting status is the exclusive status of Fair state.
%We specify these two conditions as follow:
%
%\begin{footnotesize}
%\begin{verbatim}
%rl [fair_exit] : 
%{(gstat: nFin) (switch: B1,B2,
%noFair,B4,numOfVeInTLane) 
%(lane[L]: (VI ; VS)) (v[VI]: L,halting) OCs} 
%=> {(gstat: nFin) (switch: B1,B2,
%noFair,B4,numOfVeInTLane) (lane[L]: (VI ; VS)) 
%(v[VI]: L,approaching) OCs} .
%if L == Through .
%\end{verbatim}
%\end{footnotesize}










% In Figure \ref{fully_fig}.(a), if two vehicle is approaching the merge point, vehicle on non-through lane will stop before the merge point and let 


% \verb!veInNTLane! shows non-through lane has vehicle which is approaching the merge point and \verb!numOfVeInTLane! explains number of vehicle on through lane is approaching the merge point. 


 
\section{Model Checking}
\label{sect_model}
There are three desired properties the \textit{a-merging} protocol should enjoy:
\begin{itemize}
    \item \textit{Mutual exclusion} - There is at most one vehicle crossing the merge point at any given time.
    \item \textit{Deadlock freedom} - There is no deadlock state.
    \item \textit{Starvation freedom} - If a vehicle is trying to cross the merge point, then the vehicle must eventually pass the merge point.
\end{itemize}

%We need an initial state that can cover all three states in our \textit{Revised Autonomous Vehicle Protocol for Merge Points}.
%Therefore, we put two space and three vehicles on the through lane, one space and three vehicles on the non-through lane.
%To cover all situations (three states) in the \textit{a-merging} protocol, we assume that two spaces and three vehicles on the through lane, one space and three vehicles on the non-through lane.
%The \verb!init! is shown as follows:

For the three experiments checking that the protocol enjoys the three properties mentioned above, we use the same initial state in which there are six vehicles participating. The initial state namely\verb!init! is defined as follows:
\begin{small}
\begin{verbatim}
{(gstat: nFin) 
(lane[Through]: empq) (lane[Nothrough]: empq) 
(v[1]: Through,space) (v[3]: Nothrough,space)
(v[2]: Through,space) (v[4]: Through,running)
(v[5]: Through,running) 
(v[6]: Through,running)
(v[7]: Nothrough,running) 
(v[8]: Nothrough,running)
(v[9]: Nothrough,running)
(switch: noVehicleCrossing,fullP,
noVeInNTLane,0)} .
\end{verbatim}
\end{small}

\noindent
Three vehicles run on the through lane, while three others arrive from the non-through lane.
Furthermore, we also assume that there exist three space locations.

\subsection{Mutual exclusion}

The \textit{mutual exclusion} property can be checked by the following search commmand:

\begin{small}
\begin{verbatim}
search [1] in MTAS15 : 
init =>* {(v[VI]: LN, crossing) 
(v[VI']: LT, crossing) OCs} 
            such that VI =/= VI' .
\end{verbatim}
\end{small}

\noindent
where \verb!MTAS15! is the specification of the \textit{a-merging} protocol.
The search commands tries to find a state from the initial state \verb!init! such that two vehicles are crossing the merge point.
After about 2 minutes, Maude finished without finding any such a state.
It confirms that when  there are six vehicles participating, the protocol enjoys the \textit{mutual exclusion} property.

\subsection{Deadlock freedom}
To check that there is no deadlock state, we use the following command:

\begin{small}
\begin{verbatim}
search [1] in MTAS15 : init =>! {OCs} .
\end{verbatim}
\end{small}

\noindent
%This command tries to find a state that cannot let all vehicles crossed the merge point. 
No state was found. The experiment finished after around 2 minutes.
%It does not find any such state.
%Therefore, our \textit{a-merging} protocol also enjoy the \textit{deadlock freedom} property for the \verb!init!.


\subsection{Starvation freedom}
To model check that the protocol enjoys the \textit{starvation freedom} property, we need to employ Maude LTL model checking facilities.
% define $P_{a-merging}$ and $L_{a-merging}$. 
%$P_{a-merging}$ consists of one atomic proposition \verb!Fin!.
%$L_{a-merging}$ is defined as follows:
We first define two atomic propositions namely \verb!want! and \verb!passed! which takes a vehicle IDs as their argument. 
They are defined via the following equations:

\begin{small}
\begin{verbatim}
eq {(v[VI]: L,approaching) OCs} |= 
  want(VI) = true .
eq {(v[VI]: L,crossed) OCs} |= 
  passed(VI) = true .
eq {OCs} |= PROP = false [owise] .
\end{verbatim}
\end{small}

\noindent
where \verb!PROP! is a Maude variable of atomic propositions, \verb!|=! is a Maude symbol for satisfaction relation in LTL,
%The two equations mean that for all states $s$ $\in$ $P_{a-merging}L_{a-merging}(s)$ = {\verb!Fin!} if and only if $s$ contains (\verb!gstat : fin!). 
\verb!owise! is the abbreviation of otherwise, indicating that this equation will only be applied if all of the previous equations above it cannot be applied. 
The equations say that \verb!want(VI)! holds in a state $s$ iff $s$ contains \verb!(v[VI]: L,approaching)!, which means that the vehicle \verb!VI! is trying to enter the merge point. 
Likewise, \verb!passed(VI)! holds in a state $s$ iff $s$ contains \verb!(v[VI]: L,crossed)!, which means that the vehicle \verb!VI! already passed the merge point. 

The \textit{starvation freedom} property for each vehicle is then defined by the LTL formula \verb!lofree0!  as follows:
%We also define two LTL formulas as follows:

\begin{small}
\begin{verbatim}
eq lofree0(VI) = (want(VI) |-> passed(VI)) .
\end{verbatim}
\end{small}

\noindent
The formula says that if a vehicle \verb!VI! trying to cross the merge point, it eventually will pass through the merge point.
After that, the complete \textit{starvation freedom} property is specified by the LTL formula \verb!lofree! as follows:

\begin{small}
\begin{verbatim}
eq lofree = lofree0(4) /\ lofree0(5) /\ 
				lofree0(6) /\ lofree0(7) /\ 
				lofree0(8) /\ lofree0(9).
\end{verbatim}
\end{small}

The following command is used to model check the property:
\begin{small}
\begin{verbatim}
red in MTAS15-CHECK : modelCheck(init,lofree) .
\end{verbatim}
\end{small}

\noindent
No counterexamples were found. It took about 2 minutes for Maude to complete the model checking. 
Consequently, we can conclude that the protocol enjoys the \textit{starvation freedom} property with six vehicles participating in the protocol.

%\noindent where \verb!<>! is the LTL eventually connective $\diamondsuit$.
%We run the Maude command \verb!modelCheck(init, halt)! to check whether the protocol enjoys the \textit{starvation freedom} property.
%The program returns no counterexample.
%Therefore, the \textit{a-merging} protocol enjoys the property for the \verb!init!.

 
 
\section{Related Work}
 \label{sect_Relate}
 
 An autonomous vehicle intersection control protocol called LJPL protocol has been proposed to handle the mutual exclusion property in the intersection~\cite{LimJongBeom2018Aedm}.
 The authors design two type of lanes: concurrent and conflict lanes. 
 The protocol allows that the vehicles on concurrent lanes can enter the intersection simultaneously while the vehicles on conflict lanes cannot enter the intersection simultaneously.
 The intersection in the LJPL protocol is applied to the traffic which containing eight lanes, while the \textit{merging} protocol or \textit{a-merging} protocol focuses on two lanes.
 M.N.Aung, et.al~\cite{DBLP:conf/seke/AungP019} has specified and model checked the LJPL protocol in Maude.
 They have revised the protocol to avoid the deadlock state that cannot decide the order of two vehicle on two conflict lanes have same time arrival.
 In our case, we also use Maude to model check some properties of our protocol.
  
 Vaio, et. al.~\cite{8790807} have proposed a protocol to handle the intersection which may contain 12 lanes.
 They define Conflicting Area (CA) where collisions could occur, and Cooperative Zone (CZ) where each vehicle is able to communicate with its neighbors. 
 The authors describe their problem into undirected graph where a node represents a vehicle and an edge represents a communication link between two vehicles.
 The experiment uses three autonomous vehicle to confirm their effective theoretical analysis.
 Currently, we do not find any works to use formal methods for this work.
 One piece of our future work is to specify this protocol and make it more reliable by formal methods.
 
 








\section{Conclusion}
\label{concl_sect}

We have proposed a protocol call \textit{a-merging} protocol based on the a merging protocol~\cite{10.1145/3055004.3055028}, formally specified and conducted some experiments by model checking in Maude.
Three properities such as, \textit{mutual exclusion}, \textit{deadlock} and \textit{starvation freedom}, have been model checked with the traffic containing three vehicles and spaces on both lanes.
%We have revised the \textit{Autonomous Vehicle Protocol for Merge Points} \cite{10.1145/3055004.3055028} and some case studies in our revised protocol and formal specified it in Maude.
%We also model checked our formal specification with Maude model checking facilities.
We have encountered the state explosion problem when we add more vehicle on two lanes.
In the future, we need to use theorem proving, such as CafeOBJ~\cite{DiaconescuF98}, to prove our abstract protocol enjoys some desired properties in general.


% An example of a floating figure using the graphicx package.
% Note that \label must occur AFTER (or within) \caption.
% For figures, \caption should occur after the \includegraphics.
% Note that IEEEtran v1.7 and later has special internal code that
% is designed to preserve the operation of \label within \caption
% even when the captionsoff option is in effect. However, because
% of issues like this, it may be the safest practice to put all your
% \label just after \caption rather than within \caption{}.
%
% Reminder: the "draftcls" or "draftclsnofoot", not "draft", class
% option should be used if it is desired that the figures are to be
% displayed while in draft mode.
%
%\begin{figure}[!t]
%\centering
%\includegraphics[width=2.5in]{myfigure}
% where an .eps filename suffix will be assumed under latex, 
% and a .pdf suffix will be assumed for pdflatex; or what has been declared
% via \DeclareGraphicsExtensions.
%\caption{Simulation Results}
%\label{fig_sim}
%\end{figure}

% Note that IEEE typically puts floats only at the top, even when this
% results in a large percentage of a column being occupied by floats.


% An example of a double column floating figure using two subfigures.
% (The subfig.sty package must be loaded for this to work.)
% The subfigure \label commands are set within each subfloat command, the
% \label for the overall figure must come after \caption.
% \hfil must be used as a separator to get equal spacing.
% The subfigure.sty package works much the same way, except \subfigure is
% used instead of \subfloat.
%
%\begin{figure*}[!t]
%\centerline{\subfloat[Case I]\includegraphics[width=2.5in]{subfigcase1}%
%\label{fig_first_case}}
%\hfil
%\subfloat[Case II]{\includegraphics[width=2.5in]{subfigcase2}%
%\label{fig_second_case}}}
%\caption{Simulation results}
%\label{fig_sim}
%\end{figure*}
%
% Note that often IEEE papers with subfigures do not employ subfigure
% captions (using the optional argument to \subfloat), but instead will
% reference/describe all of them (a), (b), etc., within the main caption.


% An example of a floating table. Note that, for IEEE style tables, the 
% \caption command should come BEFORE the table. Table text will default to
% \footnotesize as IEEE normally uses this smaller font for tables.
% The \label must come after \caption as always.
%
%\begin{table}[!t]
%% increase table row spacing, adjust to taste
%\renewcommand{\arraystretch}{1.3}
% if using array.sty, it might be a good idea to tweak the value of
% \extrarowheight as needed to properly center the text within the cells
%\caption{An Example of a Table}
%\label{table_example}
%\centering
%% Some packages, such as MDW tools, offer better commands for making tables
%% than the plain LaTeX2e tabular which is used here.
%\begin{tabular}{|c||c|}
%\hline
%One & Two\\
%\hline
%Three & Four\\
%\hline
%\end{tabular}
%\end{table}


% Note that IEEE does not put floats in the very first column - or typically
% anywhere on the first page for that matter. Also, in-text middle ("here")
% positioning is not used. Most IEEE journals/conferences use top floats
% exclusively. Note that, LaTeX2e, unlike IEEE journals/conferences, places
% footnotes above bottom floats. This can be corrected via the \fnbelowfloat
% command of the stfloats package.


% conference papers do not normally have an appendix


% use section* for acknowledgement
% \section*{Acknowledgment}

% The authors would like to thank...
% more thanks here


% trigger a \newpage just before the given reference
% number - used to balance the columns on the last page
% adjust value as needed - may need to be readjusted if
% the document is modified later
% \IEEEtriggeratref{8}
% The "triggered" command can be changed if desired:
% \IEEEtriggercmd{\enlargethispage{-5in}}

% references section

% can use a bibliography generated by BibTeX as a .bbl file
% BibTeX documentation can be easily obtained at:
% http://www.ctan.org/tex-archive/biblio/bibtex/contrib/doc/
% The IEEEtran BibTeX style support page is at:
% http://www.michaelshell.org/tex/ieeetran/bibtex/
\bibliographystyle{IEEEtran} \bibliography{paper}
%\bibliographystyle{IEEEtran} \bibliography{IEEEabrv,paper}
% argument is your BibTeX string definitions and bibliography database(s)
%\bibliography{IEEEabrv,../bib/paper}
%
% <OR> manually copy in the resultant .bbl file
% set second argument of \begin to the number of references
% (used to reserve space for the reference number labels box)

% \begin{thebibliography}{1}
% 
% \bibitem{IEEEhowto:kopka}
% H.~Kopka and P.~W. Daly, \emph{A Guide to \LaTeX}, 3rd~ed.\hskip 1em plus
%   0.5em minus 0.4em\relax Harlow, England: Addison-Wesley, 1999.
% 
% \end{thebibliography}

% that's all folks
\end{document}
